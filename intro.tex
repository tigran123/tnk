\thispagestyle{empty}
\bibmark{book}{\beginL\normalfont\scshape Introduction\endL}
\bibmark{oddhead}{\beginL\normalfont\scshape Introduction\endL}

\begin{center}
\Large\scshape Introduction
\end{center}

\fontsize{12}{13}\selectfont

The present edition of the Massoretic Text of the Hebrew Bible is
based on the data collected by Dr Christian David Ginsburg.

The main features of this edition are as follows:
\begin{itemize}
\item No variations, however strongly supported by the Hebrew manuscripts
and printed editions have been introduced into
the text but were rele\-ga\-ted to the margins and the footnotes.
\item While the modern divisions of chapters and verses are noted for
the sake of convenient reference, the text is arranged according 
to the ancient chapters and sectional divisions of the Massorah
and the MSS., which are thus restored.
\item I uniformly reproduce the \bibemph{Dageshed} and \bibemph{Raphed}
letters, which are found in all the best Massoretic Manuscripts, but which
have been omitted in all the current printed editions of the Hebrew Bible.
\item The ancient Massoretic chapters, called \bibemph{Sedarim}, are also
indicated in the margins against their respective places.
\item It is well known that in the printed Texts the variations called
\bibemph{Kethiv} and \bibemph{Keri} are marked by the word in the Text
(\bibemph{Kethiv}) having the vowel-points belonging to the word in the
margin (\bibemph{Keri}). This produces hybrid forms, which are a grammatical
enigma to the Hebrew student. But in this Edition the words in the Text
thus affected (\bibemph{Kethiv}) are left \bibemph{unpointed}, and in
the margin the two readings are for the first time given with their
respective vowel-points.
\item The footnotes contain the various readings of the different Standard
codices which are \bibemph{quoted in the Massorah itself}, but which
have long since perished.
\item It gives the readings of the Eastern and Western Schools against those
words which are affected by them; lists of which are preserved, and given
in the Model Codices and in certain special Manuscripts.
\item It also gives, against the affected words, the variations between
\bibemph{Ben-Asher} and \bibemph{Ben-Naph\-ta\-li}, hitherto not indicated
in the footnotes of printed editions.
These had been consigned to the end of the large Editions of the Bible which
contain the Massorah of Jacob ben Chayim.
\item It gives, for the first time, the \bibemph{entire} class of various
readings called \bibemph{Sevirin} against every word affected by them.
These \bibemph{Sevirin} in many Manuscripts are given as the substantive
textual reading, or as of equal importance with the official \bibemph{Keri}.
These readings have been collected from numerous Manuscripts.
\end{itemize}

The electronic (PDF) version of this work has the following additional
features compared to the printed version:
\begin{itemize}
\item Each chapter of the Bible can be listened to while reading, by
clicking on the chapter number (Hebrew letter) in the margin.
\item There is a bookmark for every book, every chapter within a book and
every verse within a chapter to allow quick navigation of the text using
a PDF viewer or an eBook reader.
\item The page numbers in the Table of Contents are hyperlinks which
can be followed in a PDF viewer.
\item The Biblical Genealogy is presented in the form of a PDF bookmark tree.
Following a bookmark in this tree brings up the verse describing the birth
and naming of the corresponding person.
\end{itemize}

May the \textsc{Lord} God of our fathers use this labour of love to open
the eyes of many in Israel that they may come to know the glory of our own
people and the light to lighten the Gentiles.
First and foremost, I thank the \textsc{Lord} God of all Israel and his son
our Lord Jesus Christ for graciously providing everything his slave needed
for preserving his precious words in this generation.

It is my pleasure to acknowledge the helpful contributions from the
following people (in alphabetical order by first name):
Andreas Matthias,
Anoush Yavrian,
Donald Arseneau,
Duane D. Miller,
Eric Archibald,
Eric Browning,
Ewan MacLeod,
Heiko Oberdiek,
Jonathan Melville,
Kirk Lowery,
Mario Valente,
Mark E. Shoulson,
Reinhard Kotucha,
Ricardo Shahda,
Ron Stewart,
Saul Levin,
Sebastian Rahtz,
Stefano Scaglione,
Victor Nechipurenko,
Vladimir Volovich and
Yannis Haralambous.

\begin{flushright}
\itshape
Tigran Aivazian\\
London, England\\
January 2007.
\end{flushright}
